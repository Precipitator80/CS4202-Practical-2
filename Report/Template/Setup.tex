% Language setting
\usepackage[UKenglish]{babel}
\usepackage[utf8]{inputenc}
\usepackage{csquotes}
\usepackage{graphicx} % Required for inserting images
% Set page size and margins
% Replace `letterpaper' with`a4paper' for UK/EU standard size
\usepackage[a4paper,top=2.5cm,bottom=2.5cm,left=3cm,right=3cm,marginparwidth=1.75cm]{geometry}
% ------------------------------ %
% Various useful packages
\usepackage{amsmath}
\usepackage{graphicx}
\usepackage[colorlinks=true, allcolors=blue]{hyperref}
\usepackage{multirow}
\usepackage{float}
% IEEE bibliography setup.
\usepackage[backend=biber, style=ieee, isbn=false,sortcites, maxbibnames=6, minbibnames=1]{biblatex}
\addbibresource{ref.bib} % The references.bib file in which the bibliography used should be.
\usepackage{tabularx} % Tables with dynamic columns.
\usepackage{amssymb}
\usepackage{colortbl}
\usepackage{hyperref}
\usepackage{tikz}
\usepackage[T1]{fontenc} % Extra font options such as small capital (textsc).
\usepackage{lipsum} % Lorem Ipsum Package for placeholder text.
%\usepackage{comment} % Multi-line / Block comments. Doesn't seem to work.
%\usepackage{titlesec}
%\titlespacing*{\subsection}{0pt}{0.5\baselineskip}{0.5\baselineskip}

% Enumerate tag using the alphabet instead of numbers - A.Ellett - https://tex.stackexchange.com/questions/129951/enumerate-tag-using-the-alphabet-instead-of-numbers - Accessed 15.04.2023
\usepackage{enumitem} % Better list customisation

% Subfigures
\usepackage{subfigure}
\usepackage{subcaption}

% Frames for enumerate and itemize.
\usepackage{framed}

%\usepackage{wrapfig} % Wrap figures.
% ------------------------------ %
% Trees
\usepackage[linguistics]{forest}

%\begin{center}
%\begin{forest}
%[Root
%    [L1.1
%        [L2.1\\L2.1E]
%        [L2.2
%            [L3.1\\L3.1E]
%        ]
%    ]
%    [L1.2
%        [L2.3\\L2.3E]
%        [L2.4
%            [L3.2\\L3.2E]
%            [L3.3
%                [L4.1\\L4.1E]
%                [L4.2
%                    [L5.1\\L5.1E]
%                ]
%            ]
%        ]
%    ]
%]
%\end{forest}
%\end{center}



% ------------------------------ %
% Advanced plotting, including from files.
\usepackage{pgfplots}
\pgfplotsset{width=10cm,compat=1.9}

%%% Externalisation does not seem to work.
%% Externalise the figures for faster compilation.
%\usepgfplotslibrary{external}
%\tikzexternalize
%\usetikzlibrary{external}
%\usetikzlibrary[external]
%\tikzexternalize[prefix=tikz/]
% ------------------------------ %
% Code listing - Overleaf - https://www.overleaf.com/learn/latex/Code_listing - Accessed 17.10.2023
% How to make a figure with code? - James - https://tex.stackexchange.com/questions/503533/how-to-make-a-figure-with-code - Accessed 17.10.2023
\usepackage{listings}
\usepackage{xcolor}

\definecolor{codegreen}{rgb}{0,0.6,0}
\definecolor{codegray}{rgb}{0.5,0.5,0.5}
\definecolor{codepurple}{rgb}{0.58,0,0.82}
\definecolor{backcolour}{rgb}{0.95,0.95,0.95}

\lstdefinestyle{mystyle}{
    backgroundcolor=\color{backcolour},   
    commentstyle=\color{codegreen},
    keywordstyle=\color{magenta},
    numberstyle=\tiny\color{codegray},
    stringstyle=\color{codepurple},
    basicstyle=\ttfamily\scriptsize, % Font size (i.e. \scriptsize or \footnotesize)
    breakatwhitespace=false,         
    breaklines=true,                 
    captionpos=b,                    
    keepspaces=true,                 
    numbers=left,                    
    numbersep=5pt,                  
    showspaces=false,                
    showstringspaces=false,
    showtabs=false,                  
    tabsize=2,
    aboveskip=20pt,
}

\lstset{style=mystyle}

%\noindent\begin{minipage}{\linewidth}
%\begin{lstlisting}[caption={captiontext}, label={code:labelname}, frame=single]
%\end{lstlisting}
%\end{minipage}
% ------------------------------ %
% Headers
\usepackage{fancyhdr}
\pagestyle{fancy}
\makeatletter
\lhead{\@title}
\rhead{\@author}
\setlength{\headheight}{15pt}
% ------------------------------ %
% Custom commands
% Guide text: Substitution
\newcommand{\substitution}[1]{[\textit{#1}]}

% Guide text: Optional
\newcommand{\optional}[1]{[#1]}

% Checkmarks and Crossmarks
\usepackage{pifont}
\newcommand{\cmark}{\ding{51}}%
\newcommand{\xmark}{\ding{55}}%
% ------------------------------ %
% Other References:
% set a maximum width and height for an image - Yiannis Lazarides - https://tex.stackexchange.com/questions/47245/set-a-maximum-width-and-height-for-an-image - Accessed 16.04.2023
%\begin{figure}[H]
        %\centering
        %\includegraphics[width=\textwidth, height=0.3\textheight, keepaspectratio]{ImagePath}
        %\caption{captionText}
        %\label{fig:figureName}
%\end{figure}
% ------------------------------ %
% Scrap code
%\usepackage{geometry}
%\geometry{a4paper, total={170mm,257mm}, left=5mm, top=5mm}